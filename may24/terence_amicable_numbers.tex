\documentclass[12pt]{article}
\title{Amicable numbers}
\author{Terence Wu}

\begin{document}

\maketitle

Amicable (“friendly”) numbers are a type of numerical phenomenon. If two natural
numbers, a and b, satisfy that all the proper divisors of a sum up to b and vice versa, then a and
b are an amicable pair.

The ancient Greek mathematician Pythagoras discovered the first known amicable pair:
220 and 284. Over the next 1500 years, many mathematicians dedicated themselves to
discovering amicable numbers. But, as you would imagine, these numbers were very rare.
Finding an amicable pair was like finding a needle in a haystack. Despite generations of intense
pursuit, no other pairs were found. In the sixteenth century, some even believed that there
existed only one pair of amicable numbers within natural numbers.

Then, in 1636, a second pair of amicable numbers—17296 and 18416—was discovered
by "amateur mathematician king" Fermat. Just two years later, René Descartes, known as the
father of analytic geometry, also found a pair of amicable numbers: 9437506 and 9363584. In a
span of just two years, Fermat and Descartes had broken the silence in the millennia long
search for amicable numbers, reinvigorating the search for amicable numbers.

In the years following the seventeenth century, many mathematicians joined the quest to
find new amicable numbers. Yet, finding an amicable pair was no easier than before, and still,
mathematicians remained unsuccessful. Then, out of nowhere, a thunderous breakthrough
occurred. In 1747, 39-year-old Swiss mathematician Leonhard Euler announced to the world
that he had found not one, not two, but thirty pairs of amicable numbers, which later expanded
to 60 pairs! Not only did he list a table of the pairs, but he also disclosed all his computational
processes to find them. Euler's novel approach unraveled the puzzle that had baffled humanity
for over two and a half millennia and left mathematicians astounded.

Building upon the work of their predecessors, people continued to discover numerous
pairs of amicable numbers. In 1867, 16-year-old Paganini, discovered a pair overlooked by
Euler: 1184 and 1210. In 1923, mathematicians Méray and Welter published 1095 pairs of
amicable numbers, the largest of which contained 25 digits. In the same year, Dutch
mathematician Lill discovered a pair of amicable numbers with 152 digits.

With the rise of computers, we have discovered abundances of these numbers. But we
can’t simply use computations to prove statements involving these numbers. The essence of
mathematical research lies in general principles, and until we find such a pattern regarding
amicable numbers, the journey will forever remain incomplete.

Amicable numbers are tied to number theory, a branch of mathematics that deals with
properties of numbers. Mathematicians are interested in studying amicable numbers because
they provide insights into the nature of integers and can lead to discoveries in other
mathematical fields. There still are many unanswered problems regarding amicable numbers,
such as whether or not there exist infinitely many pairs of them. Perhaps you might be able to
solve these questions!

\end{document}