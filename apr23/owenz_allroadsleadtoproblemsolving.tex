\documentclass{article}
\usepackage{graphicx}

\title{All Roads Lead to Problem Solving}
\author{Owen Zhang}

\begin{document}

\maketitle
\begin{center}
    \includegraphics[scale = 0.2]{images/allroads1.png}
\end{center}

“\textit{Life is a matter of choices, and every choice you make makes you.}” Often, making good choices relies on problem-solving skills. That’s why problem solving is ubiquitous in career and personal life—facing obstacles is recognized as unavoidable, and solving them is often necessary. As we’ll see, it permeates every career, from STEM jobs to business to fast food. Problem-solving skills are a valuable asset no matter which road you choose. And math can cultivate it.

In fields like chemistry or software programming, there is an abundance of problems to solve: proving a hypothesis and repairing code bugs are two examples. The same is true of careers such as entrepreneurship, firefighting, engineering, or playing chess. But what about factory workers? Or fast food employees? While it may look less applicable at face value, in practice, striving for productivity, promotions, and opportunities for any employee is fundamentally about problem solving. 

But work is a narrow slice of the extent that we face and solve problems. On a given day, the coffee machine breaks, the car’s AC goes clunk-clunk, the Christmas tree falls on the family rat, and your author’s SIMC article is due today... In these cases (except maybe the rat) it may not be up to you to resolve the issue, but what if it is? Practice in problem-solving may ease the strain of each scenario, whether you recognize it or not.

So, problem-solving is an important skill—great. How can it be strengthened?

To answer, we have to understand the core of solving a problem. Specifically, we need a framework for how good problem-solving really works. Constructed based on existing designs, the \textit{most basic} framework goes something like this:
\begin{enumerate}
    \item Systematically analyze the problem and build a solution
    \item Reflect.
\end{enumerate}

Nothing about this 2-step structure is magical, though the results can often be. The first component is abstract; a more common, slightly deeper framework can be found \href{https://asq.org/quality-resources/problem-solving}{here}. Theories such as \href{https://en.wikipedia.org/wiki/TRIZ}{TRIZ} and books such as \textit{How to Solve It} by George Pólya are very useful resources for step one; the specifics of them are outside the scope of this article. The key is the second component, reflection. Just like someone can make school presentations for years without growth, mindless problem-solving leads to longer plateaus of little improvement.

To this end, I believe math is an effective problem solving training tool. Solving math problems forces you to confront step 1, as math problems can act as a proxy to learn to analyze and use previous knowledge along with creativity. Since math (especially math competitions) largely involve solving math problems, the skill to do step 1 naturally progresses- faster than in some other fields with a lesser relative amount of solving problems.

Moreover, the organic process of improving can be sped up by incorporating step 2. In most cases, a feedback loop is built-in: the hundreds of competitions, thousands of books, and half a dozen websites allow you to see solutions to a problem you attempted. However, when paired with active consideration, asking such questions like “how could I have noticed that solution path?” or “did I approach this problem in an effective way?” problem solving skill rises still quicker. Note that the reflection can (and likely should) be brief: around 15 seconds can be enough. 

Problems are waiting in every domain, as intrinsic and common as water and air. For your part, try being a bit more conscious- cognizant that you’re facing a problem, aware of how you’re approaching it and reflective on the results.

Every choice you make makes you. Being better at problem solving helps you make better choices.
\end{document}