\documentclass{article}
\usepackage{graphicx}

\title{The Millenium Prize Problems}
\author{Joyce Huang}

\begin{document}

\maketitle
At the start of this millennium in 2000, the Clay Mathematics Institute chose seven unsolved math problems and aptly named them the Millennium Prize Problems. Each of them, when solved, will serve as stepping stones for countless more improvements in math. The reward for solving any one of them is one million dollars. As of the end of 2022, six still remain open. Here they are:

\begin{itemize}
    \item The Birch and Swinnerton-Dyer Conjecture is about the rational solutions of equations defining elliptic curves. Currently, it has only been proven for specific subsets of equations.
    \item The Hodge Conjecture asserts that all Hodge cycles can be expressed as rational linear combinations of algebraic cycles.
    \item The Navier-Stokes Existence and Smoothness problem aims to determine if the Navier-Stokes system of equations about fluid motion always have smooth solutions.
    \item The $P$ versus $NP$ problem resides in the field of computer science, and deals with the overlap of sets $P$ and $NP$. $P$ contains all problems where a solution can be found in polynomial time, while $NP$ contains all problems where a solution can be verified in polynomial time. Mathematicians seek to prove either that $NP$ is a subset of $P$ or that it is not.
    \item The Riemann Hypothesis states that all nontrivial complex zeros of the Riemann zeta function $\zeta(s)$ have real parts equal to $\frac12$. 
    \begin{multline*}
    \zeta(s)=\sum_{n=1}^{\infty}n^{-s} \\
    =\frac1{1^s}+\frac1{2^s}+\frac1{3^s}+\cdots
    \end{multline*}
    \begin{center}
        \footnotesize
        The Riemann zeta function.
    \end{center}
    Its solution will provide much more insight on the distribution of prime numbers.
    \item The Yang-Mills Existence and Mass Gap problem seeks to prove mathematically certain experimentally known properties of elementary particles in quantum physics, such as positive masses.
\end{itemize}
Of these Millennium Prize Problems, the only one solved as of February 2023 is the Poincar\'e conjecture. The conjecture is about spheres and manifolds in three-dimensional spaces. After many false proofs, Russian mathematician Grigori Perelman published the first correct proof of the Poincar\'e Conjecture in 2002 and 2003, simultaneously proving a related but more powerful conjecture called Thurston's geometrization conjecture.

All these open problems have been vigorously analyzed for decades and centuries by mathematicians across the world, many of whom dedicate their entire careers to the study of one particular conjecture. Of course, the million dollar reward is a nice perk, but that is the least driving motivation for these mathematicians. These problems are all critical to the field of mathematics and beyond, and when solved will impact the whole world.
\end{document}