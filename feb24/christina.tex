\documentclass{article}
\usepackage{graphicx} % Required for inserting images

\title{The Story of Christina's Sandwich}
\author{Christina Liu}
\date{February 2024}

\begin{document}

\maketitle

Yesterday night, for a makeshift dinner, I made a simple ham sandwich with two slices of bread and a slice of ham.

It looked quite tasty, actually. But there was a slight issue: with a sandwich so big, it was difficult to maintain the perfect ratio of ham and bread in each bite. To solve this dilemma, I would usually cut my sandwiches into perfectly halved slices. But I needed a knife to do that first.

Unfortunately, as I was making my way towards the table, disaster struck. I tripped on a spare room-sized machete lying on the ground, causing my sandwich to fly off my plate.The slices of food were flung to several random parts of the room as I flopped to the ground helplessly. Truly, a tragedy to rival that of Romeo and Juliet.

Normally, this would cause despair to a sandwich ratio connoisseur such as myself. However, as I live my double-life as a wacky math theorem enjoyer, I smiled in peace, knowing a simple truth: no matter where the slices landed, I would always be able to use my room-sized machete to perfectly bisect each slice, thus maintaining the perfect ratio of meat and dough.

As I swung my sword with terrifying precision, three thoughts ran through my mind. Could this be generalized to different dimensions? Were there any more practical uses for this? And finally, the most important of all, would the ham sandwich still be delicious?

Well…let’s attempt to solve these answers one at a time!

First of all, yes. Our ham-sandwich-bisection hypothesis – better known as the ham sandwich theorem – does in fact generalize to different dimensions. In other words, you can take the twenty-eight twenty-eighth-dimensional components (obviously, you need more filling choices to satiate your twenty-eight dimensions) of a twenty-eighth-dimensional sandwich, and slice it with a twenty-seventh dimension machete hyperplane into pieces of equal volume. So, even in different dimensions, the ham sandwich theorem continues to provide great results.

Equally as interesting is another fact: you can use this theorem over and over, just like L'Hôpital's Rule. Suppose we return to our three-dimensional sandwich (which is still floating in the air from our first cut). We can apply the ham sandwich theorem again to the left-side cuts and bisect each of them again in a single cut, and do the same to the right side. Thus, we end up with perfectly quadrisected pieces of perfectly proportioned sandwich.

The implications of this irl are quite interesting as well. Our original ham sandwich theorem can be expanded to a new corollary: after slicing our ham and bread into tiny little bits, we can always find a line that perfectly bisects the number of ham and bread bits between the two sides. There’s a slight issue: an odd number of bits results in the leftover one being split straight down the middle.

Think about it. This generalizes not only to our little ham sandwich. We can find a line that perfectly bisects the amount of ANYTHING in the world! Somewhere out there, there exists a line that perfectly splits the number of every single New York pizza, birthday cake, and Samsung Neo QLED 4K TV (QND95) into two halves. The possibilities are truly endless!

\end{document}