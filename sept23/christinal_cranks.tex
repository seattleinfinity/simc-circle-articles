\documentclass{article}
\usepackage{graphicx} % Required for inserting images

\title{\textbf{Tinfoil Hats in the Math Community}}
\author{Christina Liu}
\date{August 2023}

\begin{document}

\maketitle

We’ve no doubt heard all about conspiracy theorists throughout our lives in all sorts of communities. Medical professionals have to deal with anti-vaxxers, physicists battle perpetual motion believers, and astronomers grapple with flat earthers and moon landing disbelievers.

So there must be a mathematical equivalent, right?

And the answer is, yes!

If you’ve stayed long enough in the mathematical community, you’ve probably heard about open problems (aka unsolved problems) at least once in your life. As their name suggests, these problems are questions about mathematics that remain unsolved, although humans have put in much effort. Despite millenia of the world’s greatest minds working together, the answers to these ancient conundrums have managed to evade our grasp. Some seem cheekily simple, but have deflated the confidence of thousands of mathematicians.

Enter: the crank. Although they don’t have much experience in elementary mathematics, they’ve got the spirit and the confidence of a great mathematician. Seeing all those big names being unable to solve a certain problem draws their attention. But unlike other people who might cower in fear of said open problem once seeing all the failed attempts, cranks think differently. Who cares if they’ve only just started looking into math? Maybe their outside perspective will give brand new ideas on these old dilemmas! They’ll brazenly rush headlong into the question and immediately conjure up a ``simple solution,'' only that there are fatal errors in the reasoning that have slipped under their radar, rendering the entire thing utterly wrong. Contrary to seasoned mathematicians who will run through their papers time and time again to check for these sorts of little errors, cranks prefer speed over accuracy, rushing to gain widespread attention for their contributions with as little work as possible. The sheer number of cranks and the complicated messes of wildly false reasoning they produce have driven even the most mild mathematicians crazy.
And that is exactly what this essay will be about: unsolved problems and the people who ``solve'' them.

To begin our nature documentary on the mathematical crank, we must first understand the history behind their existence. Cranks have been present since ancient times, and different problems tend to draw different amounts of cranks. The three Greek problems of antiquity are particularly enticing, so much so that cranks who attempt to solve these problems have been given special names: ``circle squarers'' and ``angle trisectors.'' In ancient Greece, they would have been called ``$\tau\epsilon\tau\rho\alpha\gamma\omega\nu\iota\zeta\epsilon\iota\nu$'', or those who opted ``to occupy oneself with the quadrature.'' What a niche mathematical insult!

Perhaps the most famous example of a circle-squarer crank was Edward J Goodwin. Goodwin, a physician by trade, became fascinated with the open problem of creating a square with the same area as a circle using only a compass and a straightedge. He discovered a ``solution'' and immediately brought it forth to the government, ostentatiously presenting his ``new mathematical truth.'' Rather unfortunately for him, however, he failed to realize several major mistakes in his calculations, causing $\pi$ to equal 3.2 in his solution among a plethora of other errors. A visiting professor quickly detected the errors, narrowly preventing the bill from passing. To this day, Goodwin and his solution are still ridiculed by way of political cartoons and word of mouth.

Goodwin was not the only one to do something like this, however. A president of the Duquesne University, Reverend Jeremiah J Callahan, once claimed to have found a way to trisect an angle using only a compass and a straightedge. In the early days of his supposed discovery, he refused to share his technique with anyone, fearing that without a copyright obtained, someone could steal his solution. News outlets got a hold of the story and he and his solution quickly became a hot topic. In one interview, Callahan even boldly stood against mathematician Eric Temple Bell, who had stated the problem was proven impossible many years prior. ``It is his privilege to think as he likes…there is a solution,'' he claimed. However, when he finally obtained his copyright and published it to the world, keen-eyed mathematicians immediately noticed that Callahan had found a way to triple an angle, not trisect one. This little distinction made all the difference, and Callahan instantly became a certified mathematical crank.

These are just two famous examples who shared their claims with the public. However, there also exist many cranks who didn’t have their time to shine in the spotlight. From 1774 to 1775, the French Royal Academy of Sciences received over 150 letters from people claiming to have squared the circle, but dismissed all of them, eventually creating a rule that abolished such letters from being sent. 

All this talk about cranks really gets you thinking about how to identify one, and maybe even diagnose yourself. A disgruntled mathematician actually created a little rating guide, where you start with a score of -5 and add points as you work down the list and recognize things you do. Here’s the list for reference:

\begin{itemize}
    \item 1 point for each word in all capital letters
    \item 5 points for every statement that is clearly vacuous, logically inconsistent, or widely known to be false
    \item 10 points for each such statement that is adhered to despite careful correction
    \item 10 points for not knowing (or not using) standard mathematical notation
    \item 10 points for expressing fear that your ideas will be stolen
    \item 10 points for each new term you invent or use without properly defining it
    \item 10 points for stating that your ideas are of great financial, theoretical, or spiritual value
    \item 10 points for beginning the description of your work by saying how long you have been working on it
    \item 10 points for each favorable comparison of yourself to established experts
    \item 10 points for citing an impressive-sounding, but irrelevant, result
    \item 20 points for naming something after yourself
    \item 30 points for not knowing how or where to submit their major discovery for publication
    \item 30 points for confusing examples or heuristics with mathematical proof
    \item 40 points for claiming to have  “proof” of an important result but not knowing what established mathematicians have done on the problem
\end{itemize}
	
Today, we dove into the weird, wacky world of mathematical cranks, whose ostentatious and convoluted proof language and steadfast enthusiasm for the impossible have terrorized mathematicians for eons. But perhaps being a crank isn’t as bad as we all might think it is. Cranks undoubtedly have a great ability to dive headfirst into challenges, and perhaps with a little mathematical training, could hone their skills and eventually become better mathematicians!

\end{document}