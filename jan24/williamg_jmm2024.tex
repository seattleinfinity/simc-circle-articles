\documentclass{article}
\usepackage[utf8]{inputenc}
\usepackage{asymptote}
\usepackage{amsmath}
\usepackage{setspace}

\title{Short Encounters at JMM 2024: an Extension from Last Year}
\author{William Gvozdjak}   
\date{}

\begin{document}

\maketitle

If you've been reading the SIMC Circle for a while, you might remember Edward Yu's article \textit{Short Encounters at JMM 2023} from April 2023. Well, this year, a few other SLG members and I were able to go to the Joint Mathematics Meetings 2024 (JMM 2024), held in San Francisco, California from January 3 to January 6, 2024.

First, what is the JMM? In short, it's the world's largest mathematical gathering. It attracts people from all different stages of life--from high schoolers to seasoned faculty members and retired professors--who simply share one, critical characteristic: a passion for math. I was fortunate enough to be one of these people this year!

JMM was held in the Moscone Convention Center in downtown San Francisco. The moment I stepped in its vicinity, the first thing that hit me was just how \textit{big} it was. Obviously, it would have to be huge (there would be nearly 6000 attendees!), but just standing in the lobby and exploring around the center really revealed the size of the center. It was positively \textit{massive}: everywhere you turned, it seemed like there was another hallway filled with conference rooms, each one containing a significant number of people attentively listening to some niche math talk. If you crossed the street, you were led to yet another building, containing even more rooms. And that still wasn't it: even the hotel hosted events!

Every day, the JMM schedule was jam-packed with events. From morning to evening, there was a cornucopia of events to choose from, ranging from talks on abstract algebra to quantum computing to computer science and everything in between. Because of this, it was extremely difficult to choose where to go each day: there were just too many interesting talks to go to!

And, as Edward pointed out last year, it was amazing to just turn a corner and see some celebrity from the math community. Just walking around the booths in the main conference room, I found Cliff Stoll casually attending to one station (who you might recognize from his many videos on Klein bottles from the YouTube channel Numberphile). Many of the talks I went to were from renowned mathematicians in their corresponding fields. Even Terence Tao gave multiple talks at JMM (unfortunately, I couldn't make it to any of them, though)! 

JMM also had the weird effect of making higher math less intimidating. From the outside, the mathematical research community may seem extremely intimidating, full of geniuses, strange symbols, and impossible problems. And even though those three aspects might be true, beneath all of that, you can really see that everyone is really still human, and is simply just \textit{really, really} passionate for what they do. So don't be intimidated to start exploring higher math yourself--it's not as impossible as it might seem!

If you'd like to experience all of this yourself, I'd highly recommend attending JMM in the future. Like Edward said in his article last year, JMM 2025 is scheduled to be hosted in our beautiful hometown of Seattle. So get ready for a fun-filled weekend of passionate mathematics next January!

\end{document}
