\documentclass{article}
\usepackage{graphicx} % Required for inserting images

\title{The Discovery of ... Nothing}
\author{Christina Liu}
\date{February 2024}

\begin{document}

\maketitle

Imagine this: you’re a medieval serf in Europe, doing some devious trading with a suspicious wheat dealer. Town rules indicate that you must carve out the transaction on a board, showing the trade of three potatoes for five bundles of wheat.

But you’ve just been scammed (mwomp mwomp), so you’ve got to show that you gave the guy three potatoes, but received nothing in return.
Huh. You need to indicate the concept of nothing. But how do you pack the vast, reality-shattering concept of nothingness … into a single, little symbol??

Today, you could easily express it with, well, 0. But back in the day, you’d probably be wracking your brains with confusion, staring at this blankness that was unable to express your frustration.

0, this tiny little inconspicuous circle, hides some dark secrets indeed. Between breaking reality when put below the vinculum and bridging the uncomfortable gap between positive and negative numbers, it’s also been breaking our ancestors’ brains for eons.
Our story begins with a setup not unlike the one I explained at the start of this article, but much further back in time. We can assume that somewhere around  900 to 500 BC, symbols began to be used to notate the concept of a “placeholder zero”; zeroes that distinguish 100s from 10s.

Notably, this placeholder zero first appeared in Sumerian culture as a double wedge, before it travelled to India and northern Africa and was brought to Europe through Fibonacci. As the placeholder made its way across nations, each found a different way to notate this strange idea.

But not all was smooth sailing for the introduction of 0. While some cultures accepted the symbol and concept with no problems, a lot of cultures actually rejected the idea at first, as it had some spooky magical connotations.

Our \textit{true} zero – that is to say, the number rather than the placeholder – comes into play in fifth century AD India. In around the same time frame, the Mayans also developed the idea of zero independent of those on the other side of the world.

In present day mathematics, we do take a lot of things for granted. There are a lot of notational things – like our numerical system – that are just accepted without any thought. But as we dive further into the thought behind each of these, a whole new world of complexity opens before us.

\end{document}