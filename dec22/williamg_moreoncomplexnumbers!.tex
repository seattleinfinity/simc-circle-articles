\documentclass{article}
\usepackage{graphicx}

\title{More on Complex Numbers!}
\author{William Gvozdjak}

\begin{document}

\maketitle
If you've been following \textit{The Circle} for some time, you may recall discussions of the complex numbers in past volumes (if you don't, I highly recommend checking out Volume 2!). In this article, we'll go further into depth on the complex numbers: why are they useful, and how can they be used to derive equations and solve problems?

First, we'll discuss a quick recap of what we already know. In \textit{Special Numbers in Euler's Identity}, we learned \textbf{Euler's Formula}, which gives us that $e^{i\theta}=\cos\theta+i\sin\theta$, where $e$ is a real constant (roughly $2.71828$), $i$ is $\sqrt{-1}$, and $\theta$ is an angle. When $\theta=0$, this gives us \textbf{Euler's Identity}, proclaimed as one of the most beautiful identities in all of mathematics: $e^{i\pi}+1=0$. And in \textit{Imaginarily Complex; Really Simple}, we learned some applications of the complex numbers, such as a combinatorics problem.

Let's turn out attention to a few equations that cause students many headaches in school: the trigonometric summation formulas. The two most common ones, for $\sin$ and $\cos$, are as follows:
\begin{align*}
    \sin(\alpha+\beta)&=\sin\alpha\cos\beta+\sin\beta\cos\alpha \\
    \cos(\alpha+\beta)&=\cos\alpha\cos\beta-\sin\alpha\sin\beta.
\end{align*}
The main question is: why are these equations so obscure, and how can we possibly remember them? Surprisingly, Euler's Formula gives as an answer to this question. Let's look at Euler's Formula with $\theta=\alpha+\beta$:
\[e^{i(\alpha+\beta)}=\cos(\alpha+\beta)+i\sin(\alpha+\beta).\]
Also, using simple exponent properties, we know that this is equal to $e^{i\alpha}e^{i\beta}$. We can now apply Euler's Formula to each of the individual parts:
\[e^{i\alpha}e^{i\beta}=(\cos\alpha+i\sin\alpha)(\cos\beta+i\sin\beta).\]
Expanding and keeping in mind that $i^2=-1$, we know that this is equal to
\begin{align*}
&\cos\alpha\cos\beta-\sin\alpha\sin\beta \\&+i(\sin\alpha\cos\beta+\sin\beta\cos\alpha).
\end{align*}
Therefore, we know that
\begin{align*}
&\cos(\alpha+\beta)+i\sin(\alpha+\beta) \\
&=\cos\alpha\cos\beta-\sin\alpha\sin\beta \\
&+i(\sin\alpha\cos\beta+\sin\beta\cos\alpha).
\end{align*}
But if we have two complex numbers that are equal, then their real components must be equal, and their imaginary components must be equal. Hence,
\begin{align*}
    \cos(\alpha+\beta)&=\cos\alpha\cos\beta-\sin\alpha\sin\beta \\
    \sin(\alpha+\beta)&=\sin\alpha\cos\beta+\sin\beta\cos\alpha,
\end{align*}
exactly as we wanted!

Euler's Formula can also be used to solve geometric problems. One extremely beautiful application is 2020 AIME I Problem 8:

``A bug walks all day and sleeps all night. On the first day, it starts at point $O$, faces east, and walks a distance of $5$ units due east. Each night the bug rotates $60^\circ$ counterclockwise. Each day it walks in this new direction half as far as it walked the previous day. The bug gets arbitrarily close to the point $P$. Then $OP^2=\tfrac{m}{n}$, where $m$ and $n$ are relatively prime positive integers. Find $m+n$.''

At first, this problem seems very difficult: one could try out the first few days and look for a pattern, but a pattern is not obvious to see. Instead, we use complex numbers.

The first important thing to realize is that a complex number is a vector. Therefore, if we add two complex numbers, we are adding the vectors that correspond to those numbers. For example, the complex number $i$ added to the complex number $1$ gives their vector sum: $1+i$.

\begin{center}
    \begin{asy}
        unitsize(2cm);
    
        draw((0, 0)--(1, 0), EndArrow);
        draw((1, 0)--(1, 1), EndArrow);
        draw((0, 0)--(1, 1), red, EndArrow);

        label("$1$", (0.5, 0), dir(270));
        label("$i$", (1, 0.5), dir(0));
        label("$1+i$", (0.5, 0.5), dir(135));
    \end{asy}
\end{center}

Keeping this in mind, we can try and represent each day as a separate complex number. Then, summing all of these complex numbers will give us the point that the bug approaches.

Now, how do we find the complex number for each day? If we look close at Euler's Formula, we can see that the complex number $ae^{i\theta}$ is the vector starting from the origin and ending at the point $a$ away from the origin and $\theta$ counterclockwise from the $x$-axis:

\begin{center}
    \begin{asy}
        unitsize(1.25cm);

        draw((-1.5, 0)--(1.5, 0), EndArrow);
        draw((0, -1.5)--(0, 1.5), EndArrow);

        draw((0, 0)--1.25*dir(60), EndArrow);
        label("$2e^{i60^\circ}$", 1.25*dir(60), dir(15));
    \end{asy}
\end{center}

Therefore, the first day can be represented as the complex number $5e^{i0^\circ}$: it walks east ($0^\circ$ counterclockwise from the $x$-axis) for $5$ units (it walks to a distance of $5$ from the origin). Then, the next day, the bug turns $60^\circ$ counterclockwise and walks half the distance of the first day, giving us the complex number $\frac{5}{2}e^{i60^\circ}$. The same thing occurs the day after that, giving us $\frac{5}{2^2}e^{i(60\cdot2)^\circ}$. Continuing this and summing all of the resulting complex numbers gives us a final series of
\[5e^{i0^\circ}+\frac{5}{2}e^{i60^\circ}+\frac{5}{2^2}e^{i(60\cdot2)^\circ}+\dots.\]
But this is just an infinite geometric series with first term $5e^{i0^\circ}$ and common ratio $\frac{1}{2}e^{i60^\circ}$! Therefore, we can use the formula for the sum of an infinite geometric series $\frac{a}{1-r}$. Using this formula and Euler's Formula to simplify gives us a final position of
\begin{align*}
\frac{5e^{i0^\circ}}{1-\frac{1}{2}e^{i60^\circ}}&=\frac{5}{1-\frac{1}{2}\left(\frac{1}{2}+i\frac{\sqrt{3}}{2}\right)} \\
&=\frac{20}{3-i\sqrt{3}} \\
&=5+\frac{5\sqrt{3}}{3}i.
\end{align*}
Now, we're looking for the square of the distance between this point and the origin, so we want the squared magnitude of the above complex number. This is just
\[5^2+\left(\frac{5\sqrt{3}}{3}\right)^2=25+\frac{25}{3}=\frac{100}{3}.\]
This gives us a final answer of $100+3=\boxed{103}$.

In conclusion, Euler's Formula and complex numbers, when applied in the right context, can be used to greatly simplify computations and strategies. The next time you're working with trigonometric expressions or confusing geometry problems, try complex numbers!
\end{document}