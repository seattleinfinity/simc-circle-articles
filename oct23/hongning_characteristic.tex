\documentclass{article}
\usepackage{graphicx} % Required for inserting images
\usepackage[utf8]{inputenc}
\usepackage{amsmath}
\title{Characteristic polynomials}
\author{Hongning Wang}
\date{October 2023}

\begin{document}

\maketitle

\section{Article}
While solving problem $20$ on this year's AMC $12$A, I, along with many others quickly derived the recurrence relation $c_n = 2c_{n-1}+n-2$, $c_1 = 1$ where $c_n$ represents the sum of the elements of the $n$th row. However, the question did not just ask for a recursion; it requested the value of $c_{2023} \pmod{10}.$ While my friends guessed the solution, or engineers inducted, I proceeded with a standard method: Characteristic Polynomials.

To start off, we require some big words that can be skimmed in a first reading. The main idea of this approach is that recurrences of the form $c_n = A\cdot c_{n-1} + B \cdot c_{n-2} + C \cdot c_{n-3} \dots + f(n)$ for constants $A, B, C, \dots$ are satisfied by exponential solutions when $f(n) = 0$. Also, all these recurrence relations obey the superposition principle. In other words, if linearly independent $b_1, b_2, b_3 \dots b_k$ are particular solutions to $c_n$, then the general form of $c_n$ is a linear combination of $b_1, b_2, b_3 \dots b_k$.

With the complex jargon out of the way, let's first handle the homogeneous case, where $f(n) = 0.$ 

This is best explained through example. Let's derive the closed form equation of the recurrence $f_n = 8f_{n-1} + 9f_{n-2}$ where $f_1 = 4$ and $f_2 = 32$, a key step in problem $4$ from the $2018$ USAJMO. Knowing that the particular solutions are exponential, we guess the solution to be of the form $f_n = \lambda^n$ for constant $\lambda.$ Now, the recurrence becomes \[\lambda^n = 8\lambda^{n-1} + 9\lambda^{n-2} \Longleftrightarrow \lambda^2 - 8 \lambda - 9 = 0\]. Accordingly, the recurrence is satisfied when $\lambda = 9, -1$ or equivalently when $f_n = 9^n$ and when $f_n = (-1)^n.$ By the superposition principle, the general solution is $f_n = A \cdot 9^n + B \cdot (-1)^n.$ The values of $A$ and $B$ may be found from the initial conditions, and are left as an exercise to the reader. 

Indeed, the general procedure for a homogeneous recurrence is to set the solution as $f_n = \lambda^n$ and apply the superposition principle with the initial cases. 

Now, what happens if $f(n)$ is not zero, but a polynomial? Certainly, it is sometimes possible to use algebraic tricks to create a homogeneous relation. However, a sneaky method is to identify a polynomial solution to the recurrence. This solution would be the only non-exponential solution to the recurrence. Hence, by the superposition principle, the general solution is a linear combination of the homogeneous solution (pretend $f(n) = 0$) and this polynomial solution.

As an example, let's return to the initial recurrence $c_n = 2c_{n-1} + n-2.$ If we were to use algebraic manipulation, we must eliminate $n-2$ from the recurrence. For instance, we may apply 
\begin{align}
    c_{n-1} &= 2c_{n-2} + n-3 \\
    c_{n} &= 2c_{n-1} + n-2 \\
    c_{n+1} &= 2c_{n} + n-1 
\end{align}
Now, $(3) + (1) - 2 \cdot (2)$ yields $c_{n+1}-4c_{n}+5c_{n-1}-2c_{n-2} = 0$ which is homogeneous. Setting $c_n = \lambda^n$ yields $\lambda^3 - 4\lambda^2 + 5\lambda -2 = 0 \Longleftrightarrow (\lambda - 2)(\lambda-1)^2 = 0$, so $\lambda = 1,1,2$. With the double root (triple, quadruple, etc. roots work similarly), we invoke the general theory of recurrence relations to say that the particular solutions are $n(1)^n$ and $1^n$; the other particular solution is $2^n.$ As such, $c_n = A2^n + Bn(1)^n + C1^n$. Solving for the specific constants is left to the reader. 

Finally, we conclude by presenting an application of the sneakier method. We note that the degree of any polynomial solution must be less than or equal to $1$; otherwise, the degree of the RHS exceeds that of the LHS. Therefore, if the polynomial solution is $c_n = Dn + E$, we may obtain $D = -1, E = 0.$ 

For the homogeneous version, we have $c_n = 2c_{n-1}$ which solves to the particular case of $c_n = 2^n$. The general solution is therefore $c_n = A\cdot 2^n + B(-n).$ Solving for the constants is left to the reader.

Curiously enough, the process to solve linear differential equations is almost identical to the methods presented in this article. 

\end{document}
