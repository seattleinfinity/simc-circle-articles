\documentclass{article}
\usepackage{graphicx} % Required for inserting images

\title{X}
\author{Christina Liu}
\date{November 2023}

%% \blurb{X. We see it everywhere. But where did it come from?}

\begin{document}

\maketitle
Whether you take Prealgebra or Multi-variable Calculus, there is no doubt an important letter that has been the cornerstone of your mathematical journey thus far. A symbol that has been by your side since as long as you can remember. And that letter is ... x.

X is everywhere. It’s many people’s variable of choice when it comes to algebra, the name of one of the two axes of the coordinate plane, the hook of the title of the X-Files and the X-Men, and omnipresent in Elon Musk’s companies (e.g. SpaceX and the rebrand of a certain popular bird-based application). This letter is inextricably tied to an alluring air of mystery, which fits perfectly into the unknown nature of variables. But how did it come to represent this? How did one of the least-used letters in the English alphabet become a hallmark in mathematical notation worldwide?

The answer is, we don’t really know. But we can go through its history, and try to find out.

The story of X begins with the story of the discovery of algebra. Back in ancient times, people would write math using rhetoric, in other words, purely using words. Due to a lack of symbolic notation, people would utilize all sorts of creative solutions to indicate the existence of certain unknowns. The Egyptian scribe Ahmes used hieroglyphics meaning ``heap'' or ``mass,'' and the Babylonians used several words pertaining to different geometrical dimensions, like height or width, even if the problem had nothing to do with geometry.

Writing these out over and over again must’ve gotten tiring, because over time abbreviations began to develop. In Alexandria, mathematicians used a symbol that looked like an ``s,'' and the Indian mathematician Brahmagupta used the first syllables of different colors to represent different variables. But still, we don’t see our little cross popping up anywhere…

And that’s because this is where the waters get murky.

Some believe that ``x'' was coined as a result of Spanish scholars being unable to translate certain Arabic sounds, and that the ``sh'' sound was replaced with the Greek symbol chi ($\chi$). As these works were then translated into Latin, the $\chi$s were then replaced with the regular Latin ``x''. Seems a wonderful and completely logical explanation, right? Well, when translating from one language to another, people usually not only look at the pronunciation of certain things but also more importantly their meaning. This has led some to believe that the links between each transliteration are questionable. However, this is a theory backed up by a lot of professors, so it’s all up to you to determine the validity of this argument.

Another popular theory attributes the success of x to mathematician and philosopher René Descartes, who you might remember as the guy who created the Cartesian plane. In his magnum opus, \textit{La Géométrie}, he referred to quantities using lowercase letters, of which the most frequent was x. It’s speculated that this was because the x letter blocks were the least used and therefore would make printing easier. He also tended to use letters at the end of the alphabet (x, y, z) to mark unknowns, but it remains a mystery as to what exactly led him to use ``x'' in his nomenclature for the book.

So, in the end, there doesn’t really seem to be anything super special about x. But although we don’t know exactly why or how this symbol rose to power, it still undoubtedly is one of the coolest letters to choose. Truly, nothing is more satisfying than ending off an algebra problem with two quick slashes of ink before setting down your pen and admiring the simplicity of a once-complex concept.
\end{document}